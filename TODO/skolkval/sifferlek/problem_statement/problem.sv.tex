\problemname{Sifferlek}
Termen digital root för heltalet $N$ innebär att man upprepade gånger summerar talets
ingående siffror tills summan blir $\le 10$. Till exempel för talet 34783 får man
$$3 + 4 + 7 + 8 + 3 = 25$$
$$2+5=7$$
Talet 34783 har digital root 7.
På liknande sätt definieras multiplicative digital root för heltalet N , där man istället
upprepade gånger multiplicerar ingående siffror tills produkten blir < 10. Till exempel
för talet 34783 får man
$$3 \cdot 4 \cdot 7 \cdot 8 \cdot 3 = 2016$$
$$2 \cdot 0 \cdot 1 \cdot 6 = 0$$
Talet 34783 har multiplicative digital root 0. Hos talet 34783 är alltså multiplicative
digital root och digital root olika.
Skriv ett program som tar reda på hur många tal i ett givet intervall som har sam-
ma digital root som multiplicative digital root. Intervallet ryms alltid mellan 1 och en
miljon.

\section*{Input}
Två heltal $1 \le A \le B \le 10^6$, det givna intervallet.

\section*{Output}
Antalet tal med samma digital root som multiplicative digital root.