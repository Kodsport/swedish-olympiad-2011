\problemname{Erdös-nummer}

Matematikern Paul Erdös är bl.a. känd för att vara den matematiker som skrivit flest matematiska publikationer
(Leonhard Euler har dock flest skrivna sidor). Erdös samarbetade väldigt mycket i sina publikationer. Ociterade
källor hävdar att han ofta var temporärt inneboende hos en kollega. Under besöken skrev han en publikation
tillsammans med kollegan och när gästfriheten tog slut flyttade Erdös vidare till nästa kollega.

Det innebär stor prestige att ha skrivit en publikation med Erdös. De som har skrivit en publikation med Erdös
definieras ha Erdös-nummer 1. Ifall man skrivit en publikation med en person med Erdös-nummer 1 så får man Erdös-nummer 2 o.s.v.

Givet en lista över publikationer, skriv ut en lista med alla inblandade matematikers Erdös-nummer.

\section{Indata}
Den första raden innehåller två heltal $N$ och $M$ ($1 \le N \le 5\,000$, $1 \le M \le 40\,000$), antalet
matematiker respektive antalet publikationer.

Därefter följer $M$ rader som vardera beskriver en publikation. Varje rad börjar med ett heltal,
antalet matematiker som samarbetade i publikationen. Sedan följer matematikernas namn (mellanslags-separerade),
ordningen på matematikerna spelar ingen roll. Matematikernas namn består av 1 till 20 stora bokstäver, $A - Z$.
Inga olika matematiker har samma namn. Erdös heter \texttt{ERDOS}. I varje publikation kommer det vara minst 2
och högst 5 matematiker. Du kan anta att Erdös-numret är ändligt för alla inblandade matematiker.

\section{Utdata}
Skriv ut en nyrads-separerad lista med matematiker och deras Erdös-nummer. \texttt{ERDOS} ska vara med i denna
lista, hans Erdös-nummer är alltid 0. Listan ska vara sorterad i alfabetisk ordning. 


\section*{Poängsättning}
Din lösning kommer att testas på en mängd testfallsgrupper.
För att få poäng för en grupp så måste du klara alla testfall i gruppen.

\noindent
\begin{tabular}{| l | l | p{12cm} |}
  \hline
  \textbf{Grupp} & \textbf{Poäng} & \textbf{Gränser} \\ \hline
  $1$    & $40$          & $M \leq 5000$  \\ \hline
  $2$    & $60$          & Inga ytterligare begränsningar.  \\ \hline
\end{tabular}
